Innym podej�ciem rozpoznawania twarzy ni� EigenFace jest algorytm FisherFace. Jak mogli�my zaobserwowa� w ramach poprzedniego algorytmu \linebreak jego podstawowym celem by�o zmaksymalizowanie warto�ci wariancji \linebreak w ramach naszej bazy danych (tak by pr�bki by�y znacz� r�ne \linebreak od innych), natomiast w przypadku podej�cia zaproponowanego w ramach algorytmu FisherFace za cel stawiane jest maksymalizacja �redniego dystansu \linebreak pomi�dzy r�nymi klasami oraz zminimalizowanie wariancji \linebreak wewn�trzklasowej. Sposobem, kt�ry doprowadza do uzyskania takich warunk�w jest u�ycie liniowej metody znanej pod nazw� liniowej analizy \linebreak dyskryminacyjnej (Fisher's Linear Discriminant - FLD).